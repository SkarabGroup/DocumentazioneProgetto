\documentclass[a4paper,12pt]{article}
\usepackage[utf8]{inputenc}
\usepackage[italian]{babel}
\usepackage{graphicx}
\usepackage{array}
\usepackage{tikz}
\usepackage{geometry}
\usepackage{booktabs}
\usepackage{setspace}
\usepackage{float}
\usepackage[T1]{fontenc}
\usepackage{fancyhdr}
\usepackage{helvet}
\usepackage[table]{xcolor}
\usepackage{tabularx}
\usepackage{pdfpages}
\usepackage{longtable}
\usepackage[colorlinks=true, linkcolor=black, urlcolor=blue]{hyperref}

\geometry{margin=2.5cm}
\setstretch{1.0}

\begin{document}

\newcommand{\titoloDocumento}{Verbale Interno}
\newcommand{\dataDocumento}{2025-11-17}
\newcommand{\emailGruppo}{skarabswegroup@gmail.com}
\newcommand{\redattoreDocumento}{Suar Alberto}
\newcommand{\immagine}{../templates/skarablogo_.jpg}
\pagenumbering{gobble}
\begin{titlepage}
\centering

\vspace*{\fill}

\includegraphics[width=0.4\linewidth]{\immagine}
\vspace{1cm}

{\Huge \textbf{Skarab Group}}\\[0.5cm]

\rule{0.8\textwidth}{0.5pt}\\[0.5cm]

{\LARGE \textbf{\titoloDocumento}}\\[0.5cm]

\rule{0.8\textwidth}{0.5pt}
\vspace*{\fill}
\begin{center}
    \emailGruppo
\end{center}
\vspace*{\fill}

\end{titlepage}
\newpage
\pagestyle{fancy}
\fancyhf{}
\fancyhead[C]{\textbf{Skarab Group - Anno accademico 2025/2026}}
\fancyfoot[C]{\thepage}
\pagenumbering{arabic}
\setcounter{page}{1}
\section*{Versionamento e changelog}
\begin{center}
\begin{tabular}{|>{\centering\arraybackslash}m{2.5cm}|
                >{\centering\arraybackslash}m{1.7cm}|
                >{\centering\arraybackslash}m{5.0cm}|
                >{\centering\arraybackslash}m{2.5cm}|
                >{\centering\arraybackslash}m{2.5cm}|}
\hline
\rowcolor{gray!30}
\textbf{Data Modifica} & \textbf{Versione} & \textbf{Descrizione Modifica} & \textbf{Redattore} & \textbf{Verificatore}\\ 
\hline
2025-11-17 & 1.0.0 & Prima stesura del documento & Suar Alberto & Sandu Antonio \\
\hline
\end{tabular}
\end{center}
\newpage
\tableofcontents

\newcommand{\MeetingDate}{2025-11-15}
\newcommand{\MeetingStartingTime}{09:55}
\newcommand{\MeetingEndingTime}{11:05}
\newcommand{\Location}{Discord}
\newcommand{\Chair}{Suar Alberto}
\newcommand{\Attendees}{Suar Alberto, Zago Alice, Sandu Antonio, Sgreva Andrea, Berengan Riccardo, Basso Kevin}
\newcommand{\Assenti}{Martinello Riccardo}

\fancyhead[L]{Verbale riunione}
\fancyhead[R]{\MeetingDate}
\newpage
\section{Informazioni generali}
\subsection{Apertura}
La riunione, svolta su \Location, ha avuto inizio alle \MeetingStartingTime, introdotta da \Chair, 
che ha illustrato brevemente gli obiettivi del giorno.

\subsection{Presenti e assenti}
\begin{tabular}{@{}p{0.48\textwidth} p{0.48\textwidth}@{}}
\textbf{Presenti:} & \Attendees \\
\textbf{Assenti:} & \Assenti \\
\end{tabular}

\vspace{1em}

\section{Ordine del giorno}
\begin{enumerate}
    \item Generatore automatico di documenti
    \item Diario di Bordo
    \item Analisi tecnologie proposte dall'azienda
\end{enumerate}

\vspace{1em}

\section{Svolgimento della Riunione}

                    \subsection{Punto 1 — Generatore automatico di documenti}
                    \subsubsection*{Sintesi}
                    Il team necessita di un generatore automatico di documenti per velocizzare le tempistiche di stesura dei documenti e limitare il più possibile il tempo produttivo da rendicontare per il ruolo di Responsabile, in quanto figura chiamante della stesura dei documenti

                    \subsubsection*{Decisione}
                    Suar Alberto si occuperà di definire un generatore di documenti automatico e di renderlo disponibile a tutti i membri del gruppo
                   

                    \subsection{Punto 2 — Diario di Bordo}
                    \subsubsection*{Sintesi}
                    Per la lezione di Lunedì 17 Novembre è necessario definire il terzo diario di bordo

                    \subsubsection*{Decisione}
                    Sandu Antonio si occuperà di definire il diario di bordo per la lezione di lunedì
                   

                    \subsection{Punto 3 — Analisi tecnologie proposte dall'azienda}
                    \subsubsection*{Sintesi}
                    Ogni componente del gruppo ha svolto autonomamente una prima sessione di analisi delle tecnologie proposte dalla azienda proponente, focalizzandosi sul definire i pregi e i difetti di ognuna e delle possibili alternative da presentare alla proponente

                    \subsubsection*{Decisione}
                    Il team si impegna a continuare questa analisi, cercando di spostare l'attenzione su esercizi di potenziamento per comprendere meglio lnologie proposte e stabilire delle strutture predefinite per salvare i link e le note relative ad ogni argomento
                   


\section{Azioni e responsabilità}
\begin{longtable}{p{0.42\textwidth} p{0.22\textwidth} p{0.16\textwidth} p{0.16\textwidth}}
\toprule
\textbf{Azione} & \textbf{Responsabile} & \textbf{Scadenza} & \textbf{Stato} \\
\midrule
\endfirsthead
\toprule
\textbf{Azione} & \textbf{Responsabile} & \textbf{Scadenza} & \textbf{Stato} \\
\midrule
\endhead
Definizione template & Suar Alberto | Berengan Riccardo & 2025-11-21 & In corso \\ \midrule
Analisi delle tecnologie & SkarabGroup & 2025-11-24 & In corso \\ \midrule
Definizione primi requisiti & SkarabGroup & 2025-11-21 & In corso \\ \midrule
\bottomrule
\end{longtable}
\section*{Chiusura}
La riunione si è conclusa alle \MeetingEndingTime.

\vspace{2em}

\noindent\begin{tabular}{@{}p{0.48\textwidth}@{}}
  \textbf{Presiede:} \\
  \\[2.5em]
  \rule{0.9\linewidth}{0.6pt} \\
  \Chair \\
\end{tabular}

\vfill
\hrulefill

\end{document}